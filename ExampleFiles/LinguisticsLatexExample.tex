\documentclass[11pt,letterpaper, fleqn]{article}
\usepackage[utf8]{inputenc}
\usepackage[left=2.5cm,right=2.5cm,top=3.5cm,bottom=2.5cm]{geometry}
\usepackage{parskip}
\usepackage{fancyhdr} % Fancy header/footer
\usepackage{lastpage} % Part of fancy footer
\usepackage{enumitem} % Fancy enumerated lists
\usepackage{multicol} % Multiple columns
\usepackage{amsmath} % Equations
\usepackage{amssymb} % More equations
\usepackage[linguistics]{forest} % Trees
\usepackage{tipa} % IPA
\usepackage{expex} % Linguistic examples & glosses
\usepackage{stmaryrd} % Evaluation function brackets
\usepackage{colortbl} % Shaded cells in tableaux
\usepackage{setspace} % 1.5 (or other) line spacing
\onehalfspacing % Remove this if you don't want 1.5 line spacing

% Some custom commands to make trees easier
\newcommand{\feat}[1]{\ensuremath{\left[ \begin{smallmatrix}\textrm{#1}\end{smallmatrix} \right]}}
\newcommand{\underfeat}[2]{\ensuremath{\underset{\feat{#2}}{\textrm{#1}}}}
\newcommand{\pr}{$^\prime$}

% Expex gloss configuration to work with parskip (removes unnecessary whitespace)
\lingset{everygla={},aboveglftskip=-0.5ex,aboveexskip=1ex,belowexskip=-1ex,Everyex={\parskip=0pt}}

% This bit makes the nice page header and footer
\pagestyle{fancy}
\lhead{LING 112} % Change course code here
\rhead{Hayley Ross, \today} % Put your name here
\cfoot{\thepage\ of \pageref{LastPage}}
\setlength{\headheight}{14pt}

\title{Syntactic Theory I\\Assignment 3} % Change the title of the assignment
\author{Hayley Ross} % And put your name here again

\begin{document}

\maketitle
\thispagestyle{fancy}

I discussed this assignment with...

% Use \section* if you don't want section numbering
% Use \section if you do
\section{Examples and Glosses}

Consider the following sentence:

% This uses expex formatting - see http://mirrors.ibiblio.org/CTAN/macros/generic/expex/expex-doc.pdf
\ex This is a sentence \label{ex:first}
\xe
\pex~ % Use the tilde for consecutive examples to get better spacing
\a \ljudge{*} Sentence a
\a \ljudge{\#} Sentence b
\a Sentence c
\xe

Shiny glosses! I can refer to example (\ref{ex:first}) like this. %Requires you to have put a \label there.

\ex
\begingl
\gla Fische, die Fische fischen, fischen Fische, die Fische fischen. //
\glb fish.{\sc pl.nom/acc} which.{\sc pl.nom/acc} fish.{\sc pl.nom/acc} fish.{\sc 1/3.pl.pres} fish.{\sc 1/3.pl.pres} fish.{\sc pl.nom/acc} which.{\sc pl.nom/acc} fish.{\sc pl.nom/acc} fish.{\sc 1/3.pl.pres} //
\glft `Fish which fish fish fish fish which fish fish.' //
\endgl
\xe

\newpage

\section{Trees}

\singlespacing % Don't use 1.5 line spacing for trees
\scalebox{0.85}{ % Use a value < 1 here if your tree is too big to fit on the page
\begin{forest}
for tree={
	    l sep=0.2cm, % vertical
%	    s sep=0.2cm % horizontal
}
[CP
	[C\pr
		[C
			[$\varnothing$]
		]
		[TP
			[DP$_j$
				[D\pr
					[D[you, name=you]]
				]
			]
			[T\pr
				[T
					[\underfeat{T}{\textsc{pres}}[$\varnothing$]]
					[Mod$_i$[should, name=should]]
				]
				[ModP
					[Mod\pr
						[$t_i$, name=ti]
						[vP
							[$t_j$, name=tj]
							[v\pr
								[v
									[v[$\varnothing$]]
									[V$_k$[refrain, name=refrain]]
								]
								[VP
									[V\pr
										[$t_k$, name=tk]
										[PP
											[P\pr
												[P[from]]
[vP
	[vP
		[DP
			[D\pr
				[D[$PRO$]]
			]
		]
		[v\pr
			[v
				[v[$\varnothing$]]
				[V$_\ell$[teasing, name=teasing]]
			]
			[VP
				[V\pr
					[$t_\ell$, name=tl]
					[DP
						[D\pr
							[D[a]]
							[NP
								[AdjP
									[Adj\pr
										[Adj[nice]]
									]
								]
								[N\pr
									[N[man]]
								]
							]
						]
					]
				]
			]
		]
	]
	[PP
		[P\pr
			[P[like]]
			[DP
				[D\pr
					[D[that]]
				]
			]
		]
	]
]
											]
										]
									]
								]
							]
						]
					]
				]
			]
		]
	]
]
% Name all the nodes that you want to draw arrows from, then use the names here
\draw[->] (ti) to[out=south, in=south] (should);
\draw[->] (tj) to[out=south west, in=south] (you);
\draw[->] (tk) to[out=south, in=south] (refrain);
\draw[->] (tl) to[out=south, in=south] (teasing);
\end{forest}
}
\onehalfspacing

\section{IPA}

% For a cheatsheet see https://jon.dehdari.org/tutorials/tipachart_mod.pdf
[\textipa{Neta}]

LANGUAGE, Loniu \\
%INVENTORY,p,pʷ,t,tʃ,k,s,h,i,u,m,mʷ,n,ɲ,ŋ,e,o,l,ɛ,ɔ,r,a,w,j, \\
INVENTORY, \textipa{p,p\super w,t,tS,k,s,h,i,u,m,m\super w,n,\textltailn,N,e,o,l,E,O,r,a,w,j} \\
%PATTERN,Trigger,/a/ → [ɛ] / \underline{\quad}X[o, ɔ] (in nouns) \\
PATTERN, Trigger, \textipa{/a/ → [E] / \underline{\quad}}X\textipa{[o, O]} (in nouns) \\
%CLASS,p,t,tʃ,s,l,r,j
CLASS, \textipa{p,t,tS,s,l,r,j}

\section{Tableaux and Rules}

\begin{tabular}{| r || c | c | c |}
\hline
/cat-z/ & Agree & $\textsc{Ident-IO}(\textrm{voice})$ & *\textipa{\r*{\*C}} \\ \hline \hline
[cats] & & * & \\ \hline
[cadz] & & * & *!  \\ \hline
[catz] & *! & \cellcolor{gray!50} * & \cellcolor{gray!50} *  \\ \hline
[cads] & *! & \cellcolor{gray!50} * & \cellcolor{gray!50} * \\ \hline
\end{tabular}

\textsc{Vowel Lowering before Uvulars}. Lower high vowels to mid vowels before uvular consonants.
\[ \begin{bmatrix}
\mathrm{+syl}
\end{bmatrix} \rightarrow \begin{bmatrix}
\mathrm{-high}
\end{bmatrix} / \underline{\quad} \begin{bmatrix}
\mathrm{+cons} \\
\mathrm{+dorsal} \\
\mathrm{-high}
\end{bmatrix} \]


\section{Semantics}

\textbf{Semantic rule: Exclusive Disjunction}. If $\phi$ and $\psi$ are formulas, then $\llbracket \phi~\mathtt{XOR}~\psi \rrbracket^I = 1$ if either $\llbracket \phi \rrbracket^I = 1$ and $\llbracket \psi \rrbracket^I = 0$, or $\llbracket \phi \rrbracket^I = 0$ and $\llbracket \psi \rrbracket^I = 1$. Otherwise $\llbracket \phi~\mathtt{XOR}~\psi \rrbracket^I = 0$.

Standalone equation:
\[ \mathrm{male} := \{\langle \textrm{Agnetha, 0} \rangle, \langle \textrm{Bj\"orn, 1} \rangle , \langle \textrm{Benny, 1} \rangle, \langle \textrm{Frida, 0} \rangle\} \]

Truth table:

\begin{tabular}{c | c | c}
$P$ & $Q$ & $[P \lor Q]$ \\ \hline
1 & 1 & 1 \\
1 & 0 & 1 \\
0 & 1 & 1 \\
0 & 0 & 0 \\
\end{tabular}

Multiple aligned equations: % Use the & symbol to separate the "columns", put & before each = to get the = signs nicely lined up
\begin{align*}
S_1 &= \{\{\varnothing\}, \{A\}, A\} & S_6 &= \varnothing \\
S_2 &= A & S_7 &= \{\varnothing\} \\
S_3 &= \{A\} & S_8 &= \{\{\varnothing\}\} \\
S_4 &= \{\{A\}\} & S_9 &= \{\varnothing, \{\varnothing\}\} \\
S_5 &= \{\{A\}, A\} & &
\end{align*}

\end{document}
