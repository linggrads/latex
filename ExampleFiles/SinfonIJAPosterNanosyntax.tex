% IMPORTANT: To use this .tex file, you need to clone the Gemini template in Overleaf; this comes with a bunch of extra theme files that you need to compile this document
% https://www.overleaf.com/latex/templates/gemini-poster-theme/nzpspqjryjhx

% Gemini theme
% https://github.com/anishathalye/gemini
%
% We try to keep this Overleaf template in sync with the canonical source on
% GitHub, but it's recommended that you obtain the template directly from
% GitHub to ensure that you are using the latest version.

\documentclass[final,20pt]{beamer}

% ====================
% Packages
% ====================

\usepackage[T1]{fontenc}
\usepackage{lmodern}
\usepackage[size=custom,width=130,height=86,scale=1.4]{beamerposter}
% 1.4 * 20pt = 28pt
\usetheme{gemini}
\usecolortheme{gemini}
\usepackage{graphicx}
\usepackage{booktabs}
\usepackage{tikz}
\usepackage{pgfplots}
\usepackage{expex}
\usepackage{multicol}
\usepackage[linguistics]{forest}
\usepackage[backend=biber, natbib=true, useprefix=true, style=authoryear, maxcitenames=2, url=false, giveninits=true, uniquename=init, dashed=false]{biblatex}
\addbibresource{bibliography.bib}

\AtEveryBibitem{
	\ifentrytype{book} % Specifically leave publisher in for Caha's book
	{}
	{\ifentrytype{unpublished}
		{}
		{
		\clearlist{publisher}
		\clearfield{pages}
		\clearlist{editor}
		}
	}
}

% Remove quotes as they don't render properly
\DeclareFieldFormat[inbook]{title}{#1} 
\DeclareFieldFormat[article]{title}{#1} 

\renewcommand{\textsc}[1]{{\small\uppercase{#1}}}

% ====================
% Lengths
% ====================

% If you have N columns, choose \sepwidth and \colwidth such that
% (N+1)*\sepwidth + N*\colwidth = \paperwidth
\newlength{\sepwidth}
\newlength{\colwidth}
\setlength{\sepwidth}{0.025\paperwidth}
\setlength{\colwidth}{0.3\paperwidth}

\lingset{everygla={},aboveglftskip=-0.5ex,exskip=1ex,Everyex={\parskip=0pt}}

\newcommand{\separatorcolumn}{\begin{column}{\sepwidth}\end{column}}

% ====================
% Title
% ====================

\title{Implications of the Danish definiteness alternation for concord in Nanosyntax}

\author{Hayley Ross}

\institute[shortinst]{Harvard University}

% ====================
% Body
% ====================

\begin{document}

\begin{frame}[t]
\begin{columns}[t]
\separatorcolumn

\begin{column}{\colwidth}

\begin{alertblock}{Overview}

In a nutshell, this poster makes the following arguments:

\begin{itemize}
    \item Multiple Merge \citep{caha2019case}, the highly restrictive state of the art in Nanosyntax for handling concord, cannot handle the Danish definiteness alternation
    \item We can handle the data (including its structural allomorphy) quite elegantly if we use the less restrictive formulation of prefix workspace closure in \citet{starke2018complex} and the substantially less restrictive  copying agreement mechanism from \citet{taraldsen2010nanosyntax}
    \item Nanosyntax's prediction of “last resort” prefixes is borne out
    \item We need to find a balance between restrictiveness and expressiveness for Nanosyntax, e.g.~by revising Multiple Merge to be more flexible (but less flexible than \citealp{taraldsen2010nanosyntax})
\end{itemize}

\end{alertblock}

\begin{block}{Nanosyntax: prefixes, suffixes and Multiple Merge}
  
    Nanosyntax is characterized by two claims:

    \begin{itemize}
    \item There is a universal merge order of functional features, and a succinct universal algorithm for how they merge
    \item Lexical entries spell out phrases (trees of features)
    % \item Language variation derives from the availability of lexical entries, which spell out entire phrases (trees of features)
    \end{itemize}
    
    % We will investigate the following predictions and mechanisms:
    To analyze the Danish data, we will draw on the following mechanisms:
    
    \begin{itemize}
        % \item There is a universal merge order of functional features, and a  universal algorithm for how they merge
        % \item Lexical entries spell out phrases (trees of features)
        \item Suffixes are default, prefixes are a “last resort” \citep{starke2018complex}
        \item Disagreement on limits to prefix construction, specifically prefix workspace closure: \\
            \begin{itemize}
            \item Immediate workspace closure \citep{caha2019case}
            \item Late closure \citep{starke2018complex}
            \item Allow multiple features but only single morphemes \citep{caha2019fine}
            \end{itemize}
        \citep{caha2019case, starke2018complex, caha2019fine}
        \item Multiple Merge \citep{caha2019case} allows concord (multiple expression of a feature) and multi-morpheme prefixes
        \item Crucially, Multiple Merge allows a concord feature F (case, gender, \ldots) to “skip” an intervening adjective in the configuration [F [A N]] and be expressed on the noun N
 %       \item Alternative: agreement by copying arbitrary sets of features \citep{taraldsen2010nanosyntax}
    \end{itemize}
    
    Note that in Nanosyntax, any material that is generated on the left (in specifier position) is considered a prefix, even if it is not a traditional affix.

\end{block}

\end{column}

\separatorcolumn

\begin{column}{\colwidth}

\begin{block}{The Danish definiteness alternation}

    Danish noun phrases show structural allomorphy between a definiteness suffix and a freestanding definiteness marker, depending on the presence of an adjective \citep{hankamer2018structure}. We also observe concord between the noun and definiteness marker:

    % TODO textsc not working
    \begin{multicols}{2}
        \pex
        \a
        \begingl
        \gla kant-\textbf{en} //
        \glb edge-\textsc{def} //
        \glft `the edge' //
        \endgl \label{ex:kanten}
        \a \ljudge{*}
        \begingl
        \gla  den kant //
        \glb \textsc{def} edge //
        \glft $\approx$ `the edge' //
        \endgl
        \xe
        \pex~
        \a \ljudge{*}
        \begingl
        \gla skarpe kant-en //
        \glb sharp edge-\textsc{def} //
        \glft $\approx$ `the sharp edge' //
        \endgl
        \a
        \begingl
        \gla \textbf{den} skarpe kant //
        \glb \textsc{def} sharp edge //
        \glft `the sharp edge' //
        \endgl \label{ex:den-skarpe-kant}
        \xe
    \end{multicols}
    
    % \pex~ Restrictive and non-restrictive relative clauses
    % \a
    % \begingl
    % \gla \textbf{den} \textbf{stol} som jeg sad p\r{a} //
    % \glb \textsc{def} chair that I sat on //
    % \glft `the chair that I sat on' [restrictive only] //
    % \endgl
    % \a
    % \begingl
    % \gla \textbf{stol-en} som jeg sad p\r{a} //
    % \glb chair-\textsc{def} that I sat on //
    % \glft `the chair, which I sat on' [non-restrictive] //
    % \endgl
    % \xe
    
    % \citet{hankamer2018structure} show that the allomorphy must be structural and not linear, because the same effect occurs for restrictive relative clauses which occur to the right of the noun. We also observe concord between the noun gender (common or neuter) and the definiteness marker: 
    
    \begin{multicols}{2}
        \pex~
        \a
        \begingl
        \gla kant-\textbf{en} //
        \glb edge-\textsc{def.sg.c} //
        \glft `the edge' //
        \endgl
        \a
        \begingl
        \gla \textbf{den} skarpe kant //
        \glb \textsc{def.sg.c} sharp edge //
        \glft `the sharp edge' //
        \endgl
        \xe
        \pex~
        \a
        \begingl
        \gla hus-\textbf{et} //
        \glb house-\textsc{def.sg.n} //
        \glft `the house' //
        \endgl \label{ex:huset}
        \a
        \begingl
        \gla \textbf{det} store hus //
        \glb \textsc{def.sg.n} big house //
        \glft `the big house' //
        \endgl \label{ex:det-store-hus}
        \xe
    \end{multicols}
    
    {\small
    Glosses: \textsc{def} = definite, \textsc{c} = common gender, \textsc{n} = neuter gender
    }
    
    Note the commonality of form between \emph{-en/den} and \emph{-et/det}: we would like to explain this by analysing this as \emph{d-en/-et} (a multi-morpheme “prefix”).

\end{block}

\begin{block}{Problems with Multiple Merge}

Given a derivation of \emph{kant-en} with \emph{-en} footed in some feature F, Multiple Merge only triggers if F is merged before the adjective but D\textsc{ef} is merged after (otherwise, we merge regularly and get \emph{[skarpe kant]-en}). However, because Multiple Merge can skip the adjective, we still get \emph{skarpe [kant-en]}.

\vspace*{1em}
    \begin{columns}[t]
    \hspace*{1.8em}
    
    \column{0.25\textwidth}
    \phantom{“Regular” Merge:}
    
    \scalebox{0.69}{
    \begin{forest}
    [D\textsc{ef}P,s sep=1.5em
    	[NP,tikz={\node [draw,ellipse,inner sep=0em,fit to=tree] {};}
    		[$\sqrt{}$\ldots, roof]
    	]{\node at (.south)[below=4.5em]{\emph{kant}};}
    	[D\textsc{ef}P,tikz={\node [draw,ellipse,inner sep=-0.25em,fit to=tree] {};}
    		[\ldots, roof
    			[F]
    		]
    	]{\node at (.south)[below=8em]{\emph{en}};}
    ]
    \end{forest}
    }
    
    \column{0.3\textwidth}
    “Regular” Merge:
    
    \scalebox{0.69}{
    \begin{forest}
    [D\textsc{ef}P,s sep=2em
    	[AP,s sep=1.5em
    		[AP,tikz={\node [draw,ellipse,inner sep=0em,fit to=tree] {};}
    			[$\sqrt{}$\ldots, roof]
    		]{\node at (.south)[below=4.5em]{\emph{skarpe}};}
    		[NP,tikz={\node [draw,ellipse,inner sep=0em,fit to=tree] {};}
    			[$\sqrt{}$\ldots, roof]
    		]{\node at (.south)[below=4.5em]{\emph{kant}};}
    	]
    	[D\textsc{ef}P,tikz={\node [draw,ellipse,inner sep=-0.25em,fit to=tree] {};}
    		[\ldots, roof
    			[F]
    		]
    	]{\node at (.south)[below=8em]{\emph{en}};}
    ]
    \end{forest}
    }
    
    \column{0.4\textwidth}
    Multiple Merge:
    
    \scalebox{0.69}{
    \begin{forest}
    [AP,s sep=1.5em
    	[AP,tikz={\node [draw,ellipse,inner sep=-0.25em,fit to=tree] {};}
    		[AP
    			[$\sqrt{}$\ldots, roof]
    		]
    		[D\textsc{ef}P,edge=dashed
    			[\ldots, roof]
    		]
    	]{\node at (.south)[below=8em]{\emph{skarpe}};}
    	[D\textsc{ef}P,s sep=1.5em
    		[NP,tikz={\node [draw,ellipse,inner sep=0em,fit to=tree] {};}
    			[$\sqrt{}$\ldots, roof]
    		]{\node at (.south)[below=4.5em]{\emph{kant}};}
    		[D\textsc{ef}P,tikz={\node [draw,ellipse,inner sep=-0.25em,fit to=tree] {};}
    			[\ldots, roof
    				[F]
    			]
    		]{\node at (.south)[below=8em]{\emph{en}};}
    	]
    ]
    \end{forest}
    }
    \end{columns}
    
Furthermore, without Multiple Merge, there is no way to yield \emph{d-en/et} as a multi-morpheme prefix under \citet{caha2019case}.

\end{block}


\end{column}

\separatorcolumn

\begin{column}{\colwidth}



\begin{block}{Definiteness alternation with late workspace closure}

We decompose definiteness into two heads, provisionally D1 and D2. 
% We posit that the AP is inserted just below D1. 
The noun spells out D1 unless an AP intervenes, which instead triggers building the prefix \emph{d-en} – provided we assume late prefix workspace closure \citep{starke2018complex}, which allows D2P to suffix to D1P while still in the prefix workspace.

\vspace*{0.25em}
    \begin{columns}[t]
    \hspace*{1.8em}
    
    \column{0.5\textwidth}
    \scalebox{0.69}{
    \begin{forest}
    [D2P,s sep=3.5em
    	[D1P,tikz={\node [draw,ellipse,inner sep=-0.5em,minimum width=8.5em,minimum height=16em,yshift=-9em,xshift=-1.5em,rotate=17] {};}
    		[D1]
    		[I\textsc{nd}P
    			[I\textsc{nd}]
    			[C\textsc{lass}P
    				[C\textsc{lass}]
    				[NP
    					[$\sqrt{}$\ldots, roof]
    				]
    			]
    		]
    	]{\node at (.south)[below=12em]{\emph{kant}};}
    	[D2P,tikz={\node [draw,ellipse,inner sep=-0.25em,fit to=tree] {};}
    		[D2]
    	]{\node at (.south east)[below=4em]{\emph{en}};}
    ]
    \end{forest}
    }
    \column{0.5\textwidth}
    \scalebox{0.69}{
    \begin{forest}
    [D2P,s sep=1.2em
    	[D2P,s sep=1.2em
    		[D1P,tikz={\node [draw,ellipse,inner sep=-0.35em,fit to=tree] {};}
    			[D1]
    			[A]
    		]{\node at (.south)[below=8ex]{\emph{d}};}
    		[D2P,tikz={\node [draw,ellipse,inner sep=-0.25em,fit to=tree] {};}
    			[D2]
    		]{\node at (.south)[below=8ex]{\emph{en}};}
    	]
    	[AP
    		[AP,tikz={\node [draw,ellipse,inner sep=-0.25em,fit to=tree] {};}
    			[$\sqrt{}$\ldots, roof]
    		]{\node at (.south)[below=9ex]{\emph{skarpe}};}
    		[I\textsc{nd}P,tikz={\node [draw,ellipse,inner sep=-0.5em,minimum width=8.5em,minimum height=12em,yshift=-10.5em,xshift=7.5em,rotate=15] {};}
    			[I\textsc{nd}]
    			[C\textsc{lass}P
    				[C\textsc{lass}]
    				[NP
    					[$\sqrt{}$\ldots, roof]
    				]
    			]
    		]{\node at (.south east)[below=19ex]{\emph{kant}};}
    	]
    ]
    \end{forest}
    }
    \end{columns}
    \vspace*{-0.5em}

\end{block}

\begin{block}{Handling concord with \citet{taraldsen2010nanosyntax}}

We interpret D1 as D\textsc{ef} and D2 as a placeholder feature A\textsc{gr}D which, using \citeauthor{taraldsen2010nanosyntax}'s copying agreement, copies C\textsc{lass}, N\textsc{eut} (if present) and I\textsc{nd}.

\hspace*{5.5em}
\scalebox{0.69}{
\begin{forest}
[I\textsc{nd}P (\textsc{AgrD}),s sep=1.5em
	[C\textsc{lass}P (\textsc{AgrD}),s sep=5.5em
		[D\textsc{ef}P,tikz={\node [draw,ellipse,inner sep=-0.75em,minimum width=9.5em,minimum height=19.5em,yshift=-13.5em,xshift=-8.5em,rotate=25] {};}
			[D\textsc{ef}]
			[I\textsc{nd}P
				[I\textsc{nd}]
				[N\textsc{eut}P
					[N\textsc{eut}]
					[C\textsc{lass}P
						[C\textsc{lass}]
						[NP
							[$\sqrt{}$\ldots, roof]
						]
					]
				]
			]
		]{\node at (.south)[below=12.25em]{\emph{hus}};}
		[C\textsc{lass}P (\textsc{AgrD}),tikz={\node [draw,ellipse,inner sep=-0.35em,fit to=tree] {};}
			[\textsc{Class}]
		]{\node at (.south)[below=9ex]{\emph{e}};}
	]
	[I\textsc{nd}P (\textsc{AgrD}),tikz={\node [draw,ellipse,inner sep=-0.75em,minimum width=9.5em,minimum height=11em,yshift=-6.5em,xshift=8em, rotate=30] {};}
		[I\textsc{nd}]
		[N\textsc{eut}P (\textsc{AgrD})
			[N\textsc{eut}]
		]
	]{\node at (.south)[below=16ex]{\emph{t}};}
]
\end{forest}
}
\vspace*{-0.5em}
    
\end{block}

\begin{block}{References}
    \vspace*{-1ex}
    % \nocite{baunaz2018nanosyntax}
    \renewcommand*{\bibfont}{\normalfont\footnotesize\selectfont}
    \printbibliography[heading=none]
  \end{block}

\end{column}

\separatorcolumn
\end{columns}
\end{frame}

\end{document}
