\documentclass{article}
\usepackage[utf8]{inputenc}
\usepackage[left=2.5cm,right=2.5cm,top=3.5cm,bottom=2.5cm]{geometry}%change margins

\usepackage{times} %times new roman font
\usepackage{gb4e}


\title{}
\author{}
\date{}

\begin{document}

Consider the following sentence

\begin{exe}
\ex\label{ex:first} This is a sentence
\ex
    \begin{xlist}
    \ex[*]{Sentence a}
    \ex[\#]{Sentence b}
    \ex[]{Sentence c}
    \end{xlist}

\end{exe}

Shiny glosses!  I can refer to example (\ref{ex:first}) like this. %Have to put \ref{} in parentheses

Example (\ref{ex:Turkish}) is a nice sentence in Turkish.


\begin{exe}
    \ex\label{ex:Turkish}
        \gll 
       Öğretmen-ler öğrenci-ler-e iki kitap ok-ut-tu.\\
       teacher-{\sc pl.nom} student-{\sc pl-acc} two book read-{\sc caus-pst}\\ %Do not forget line breaks!
        `The teacher made the students read two books..'
\end{exe}

% Consider the following sentence:

% % This uses expex formatting - see http://mirrors.ibiblio.org/CTAN/macros/generic/expex/expex-doc.pdf
% \ex This is a sentence \label{ex:first}
% \xe
% \pex~ % Use the tilde for consecutive examples to get better spacing
% \a \ljudge{*} Sentence a
% \a \ljudge{\#} Sentence b
% \a Sentence c
% \xe

% Shiny glosses! I can refer to example (\ref{ex:first}) like this. %Requires you to have put a \label there.


% Example (\ref{ex:Turkish}) is a nice sentence in Turkish.

% \ex \label{ex:Turkish}
% \begingl
% \gla Öğretmen-ler öğrenci-ler-e iki kitap ok-ut-tu. //
% \glb teacher-{\sc pl.nom} student-{\sc pl-acc} two book read-{\sc caus-pst}//
% \glft `The teacher made the students read two books.'//
% \endgl
% \xe


\end{document}
