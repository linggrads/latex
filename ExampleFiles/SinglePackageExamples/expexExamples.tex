\documentclass{article}
\usepackage[utf8]{inputenc}
\usepackage[left=2.5cm,right=2.5cm,top=3.5cm,bottom=2.5cm]{geometry}%change margins
\usepackage{parskip} %For paragraph formatting
\usepackage{times} %times new roman font

\usepackage{expex} % Linguistic examples & glosses
\lingset{everygla={},aboveglftskip=-0.5ex,aboveexskip=1ex,belowexskip=-1ex,Everyex={\parskip=0pt}} % Expex gloss configuration to work with parskip (removes unnecessary whitespace).  Also unitalicizes top line of gloss


\title{}
\author{}
\date{}

\begin{document}


Consider the following sentence:

% This uses expex formatting - see http://mirrors.ibiblio.org/CTAN/macros/generic/expex/expex-doc.pdf
\ex This is a sentence \label{ex:first}
\xe
\pex~ % Use the tilde for consecutive examples to get better spacing
\a \ljudge{*} Sentence a
\a \ljudge{\#} Sentence b
\a Sentence c
\xe

Shiny glosses! I can refer to example (\ref{ex:first}) like this. %Requires you to have put a \label there.

\ex
\begingl
\gla Fische, die Fische fischen, fischen Fische, die Fische fischen. //
\glb fish.{\sc pl.nom/acc} which.{\sc pl.nom/acc} fish.{\sc pl.nom/acc} fish.{\sc 1/3.pl.pres} fish.{\sc 1/3.pl.pres} fish.{\sc pl.nom/acc} which.{\sc pl.nom/acc} fish.{\sc pl.nom/acc} fish.{\sc 1/3.pl.pres} //
\glft `Fish which fish fish fish fish which fish fish.' //
\endgl
\xe


\end{document}
