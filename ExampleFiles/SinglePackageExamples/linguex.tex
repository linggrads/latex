\documentclass{article}
\usepackage[utf8]{inputenc}
\usepackage[left=2.5cm,right=2.5cm,top=3.5cm,bottom=2.5cm]{geometry}%change margins

\usepackage{times} %times new roman font
\usepackage{linguex}


\title{}
\author{}
\date{}

\begin{document}


Consider the following sentence:

\ex.This is a sentence \label{ex:first}\\

\ex.\a.*Sentence a %Do not put space b/w judgment symobl and the first word! 
    \b.\#Sentence b
    \b. Sentence c
    
Shiny glosses!  I can refer to example \ref{ex:first} like this. %Do not put parentheses around \ref{}

Example \ref{ex:Turkish} is a nice sentence in Turkish.

\exg.Öğretmen-ler öğrenci-ler-e iki kitap ok-ut-tu. \label{ex:Turkish}\\
       teacher-{\sc pl.nom} student-{\sc pl-acc} two book read-{\sc caus-pst}\\ %Do not forget line breaks!
        `The teacher made the students read two books..'


% Consider the following sentence

% \begin{exe}
% \ex\label{ex:first} This is a sentence
% \ex
%     \begin{xlist}
%     \ex[*]{Sentence a}
%     \ex[\#]{Sentence b}
%     \ex[]{Sentence c}
%     \end{xlist}

% \end{exe}

% Shiny glosses!  I can refer to example (\ref{ex:first}) like this. %Have to put \ref{} in parentheses

% Example (\ref{ex:Turkish}) is a nice sentence in Turkish.


% \begin{exe}
%     \ex\label{ex:Turkish}
%         \gll 
%       Öğretmen-ler öğrenci-ler-e iki kitap ok-ut-tu.\\
%       teacher-{\sc pl.nom} student-{\sc pl-acc} two book read-{\sc caus-pst}\\ %Do not forget line breaks!
%         `The teacher made the students read two books..'
% \end{exe}



\end{document}
