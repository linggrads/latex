\documentclass{beamer} %This is how you declare 
%\documentclass[handout]{beamer} % This suppresses transitions

%The following two lines will make your (You should use it with [handout] set).
    %\usepackage{pgfpages}
    %\pgfpagesuselayout{6 on 1}[a4paper]

\usepackage[utf8]{inputenc}

\usepackage{times} %times new roman font
\usepackage[linguistics]{forest} %For trees
\usepackage{xcolor} %nice colors for text
\usepackage{tipa} %for ipa
\usepackage{soul} %for striking out text

\usepackage{linguex} %glossing package
\usepackage{lipsum} % For generating dummy text

\title{How to use \texttt{beamer}}
\author{Ian Kirby}
\date{Spring 2022}

\begin{document}


%%%%%%% This is the first frame
\begin{frame}{}%The {} is where you may put the title of the frame, if you want one.

    \maketitle
    
\end{frame}



%This is the second slide.
\begin{frame}{Nice frame}

    Here's what happens when write on a slide.
    \bigskip

    

\end{frame}


%This is the third frame
\begin{frame}{Frame transitions}

When you make transition text, the PDF that you generate actually includes 

\begin{itemize}
    \item<1-> This is list of items %This will appear on the first slide and those following
    \item<2-> This an item in the list %This will appear on the second slide and those following
    \begin{itemize}
        \item<3-4> Hey look I hid this one!
        \item<4-4> A nice item
        \item<5-> Hey where'd they go?
    \end{itemize}
    \item<6-> Look, I brought a friend with me! %Both items will appear on slide 6
    \item<6-> Hi, friend! 
\end{itemize}

\onslide<7->{\ex.Look, I can also do it with a gloss!}
\onslide<8->{\ag.Sag mein-e kind-er, dass ich sie lieb-e\\
tell.\textsc{imp} my-\textsc{pl} kid-\textsc{pl} that I them love-\textsc{pres.3sg}\\
`Tell my children that I love them'
}
    
\end{frame}


%%%% This is the fourth frame

\begin{frame}{Doing cool stuff with \texttt{\textbackslash $<$onslide<\textit{n}-$>$\{\}} and \texttt{\textbackslash only$<$\textit{n}$>$\{\}\>}}

\only<1>{You can do a lot of cool stuff with \texttt{\textbackslash $<$onslide<\textit{n}-$>$\{\}} and \texttt{\textbackslash only$<$\textit{n}$>$\{\}\>}.  For example...}

\only<2>{It's gone!}

\onslide<2-5>{What if I wanted to have some part of my text transition to something else.  For example...}

    \medskip

\only<3>{We have no cats, Kathleen.}
    \medskip 

\only<4-5>{

    \textbf{We have no cats, Kathleen}.\only<5>{ \textipa{[wi: h\ae v noU k\super h\ae ts k\super h\ae Tli:n]}
    }
    }
    


\end{frame}


\begin{frame}{Here's a cool thing you can do...}


\only<1>{\begin{forest}
[VP
    [{V\\have}]
    [DP [{D\\no}]
        [NP [{N\\cats}]]]]
\end{forest}}

\only<2>{\begin{forest}
[VP
    [DP [{D\\we}]]
[V'
    [{V\\have}]
    [DP [{D\\no}]
        [NP [{N\\cats}]]]]]
\end{forest}}

\only<3>{
\begin{forest}
[IP
    [{},phantom]
    [I'
    [I]
[VP
    [DP [{D\\we}]]
[V'
    [{V\\have}]
    [DP [{D\\no}]
        [NP [{N\\cats}]]]]]]]
\end{forest}
}

\only<4>{\begin{forest}
[IP
    [DP [{D\\we}, name={y}]]
    [I'
        [I]
        [VP
            [\st{DP} [{\st{D}\\\st{we}}, name={x}]]
            [V'
                [{V\\have}]
                [DP [{D\\no}]
                    [NP [{N\\cats}]
                    ]]]]]]
\draw[->] (x.west) to[bend left] (y.south);
\end{forest}}


    
\end{frame}



\end{document}
