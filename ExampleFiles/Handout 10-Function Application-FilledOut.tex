\documentclass[11pt]{article}



\usepackage{bbding}
\usepackage{pifont}
\usepackage[top=1in,left=1in,right=1in,bottom=1in]{geometry}
\usepackage{graphicx}
\usepackage{ipa}
\usepackage{array} 
\usepackage{tree-dvips}
\usepackage{tikz-qtree}
\usepackage{multicol}
\usepackage{soul}
\usepackage{tikz-qtree-compat} 
\usepackage[utf8]{inputenc}
\usepackage{mathtools}
\usepackage{bibentry}
\usepackage{amssymb,amsmath,stmaryrd}
\usepackage{tcolorbox}
\usepackage{natbib}
\usepackage{lscape}
\usepackage{parskip}
\usepackage{mdframed}
\usepackage{tikz-qtree}
\usepackage{booktabs}
\usepackage[table,xcdraw]{xcolor}
\usepackage{fancyhdr}
\usepackage[dvipsnames]{xcolor}
\usepackage[mathscr]{euscript}
\usepackage{verbatim}
\let\euscr\mathscr \let\mathscr\relax% just so we can load this and rsfs
\usepackage[scr]{rsfso}
\newcommand{\powerset}{\raisebox{.15\baselineskip}{\Large\ensuremath{\wp}}}
\usepackage{tcolorbox}
\pagestyle{fancy}
\lhead{Ya\u{g}mur Sa\u{g} - LING 106}
\rhead{Spring 2022 - Handout 10}
\renewcommand{\headrulewidth}{0.4pt}
\newcommand{\intp}[1]{$\llbracket${#1}$\rrbracket$}
\newcommand{\xmark}{\text{\ding{55}}}

\usepackage[inference]{semantic}
\usepackage{natbib} 
\usepackage{linguex}
\setlength{\bibsep}{0pt} \relax
\setcitestyle{notesep={: },yysep={, }} \relax

\newcommand*\circled[1]{\tikz[baseline=(char.base)]{
            \node[shape=circle,draw,inner sep=2pt] (char) {#1};}}
\definecolor{shadecolor}{RGB}{150,150,150}

\tikzset{every tree node/.style={align=center,anchor=north}}

\date{}

\begin{document} 

\begin{center}
\textbf{\LARGE Executing the Fregean Program\\ Function Application}\footnote{The content of this handout draws on Coppock and Champollion (2022).}
\end{center}

\vspace{20pt} 



We will now use Lambda Calculus to translate constituents of arbitrary size, from words and phrases all the way up to the sentences themselves, into logic.

Let's go over some fundamentals: 

\begin{itemize}
\item Lambda calculus expressions translate syntactic constituents and compose in a way that mirrors the syntactic structure of the sentence.	
\item It is the job of a theory of syntax to determine what these constituents are; here will only give a toy syntax that can be replaced by more sophisticated syntactic theories without significant changes to the semantics.
\item We will now execute Frege's conjecture: there is only one way for the meanings of two subexpressions to combine to give the meaning of a complex expression: application of a function to an argument. 
\item The resulting semantic system will be an \textbf{extensional semantics}. We won't talk about \textbf{intensional semantics} in this course. 
\end{itemize}

\vspace{10pt} 
\underline{Example:}

Let's now see how \textit{\textbf{Pepe meowed}} is composed: 



\ex.  \a. \textit{\textbf{Pepe}} $\rightsquigarrow$ $p$ (type $e$)
\b. \textit{\textbf{meowed}} $\rightsquigarrow$ $\lambda x. \ Meowed(x)$ (type $\langle e,t \rangle$)
\c. \textit{\textbf{Pepe meowed}} $\rightsquigarrow$ $Meowed(p)$ (type $t$)


\ex. \leavevmode\vadjust{\vspace{-\baselineskip}}\newline \begin{tikzpicture}	
[sibling distance=20pt, level distance=40pt][scale=1]
\Tree [. {$t$}\\{$Meowed(p)$} [. {$e$}\\{$p$} [. \textit{\textbf{Pepe}} ] ] [. {$\langle e, t\rangle$}\\{$\lambda x. \ Meowed(x)$} [.  \textit{\textbf{meowed}} ] ]]
\end{tikzpicture}


$\llbracket Meowed(p)\rrbracket^{M,g} \rightsquigarrow$  1 iff Pepe meowed in M. 



Assuming that Pepe meowed in $M$: 

\ex. $\llbracket Meowed(p)\rrbracket^{M,g} \rightsquigarrow$  1 

\newpage
Now, let's translate a well-behaved `fragment' of English, consisting of the following ingredients: 

\begin{enumerate}
\item a specification of our formal representation language, with syntactic and semantic rules, 
\item a specification of the syntax of the English expressions we cover, 
\item a list of lexical entries,
\item a list of composition rules
\end{enumerate}


\vspace{10pt}

We already have the first one: L$_\lambda$. Here are some examples for 2 and 3: 



\begin{multicols}{2}
\includegraphics[scale=.6]{syntax.png}

\includegraphics[scale=.6]{lexicon.png}
\end{multicols}


Here are our composition rules: 






\begin{center}
\includegraphics[scale=.6]{compositionrule.png}
\end{center}

\begin{center}
\includegraphics[scale=.6]{compositionrule2.png}
\end{center}

\newpage

\begin{multicols}{2}

Here is `Agnetha smiled': \\

\begin{tikzpicture}	
[sibling distance=20pt, level distance=60pt][scale=1]
\Tree [. S\\{$t$}\\{$Smiled(a)$} [. DP\\{$e$}\\{$a$} [. D\\{$e$}\\{$a$} [. \textbf{\textit{Agnetha}} ]  ]] [. VP\\{$\langle e, t\rangle$}\\{$\lambda x. \ Smiled(x)$} [. V\\{$\langle e, t\rangle$}\\{$\lambda x. \ Smiled(x)$}  [. \textbf{\textit{smiled}} ]]]]
\end{tikzpicture}

\columnbreak

Here is `Agnetha loves Bj\"{o}rn': \\

\begin{tikzpicture}	
[sibling distance=20pt, level distance=60pt][scale=1]
\Tree [. S\\{$t$}\\{$Loves(a, b)$} [. DP\\{$e$}\\{$a$} [. \textbf{\textit{Agnetha}}   ]] [. VP\\{$\langle e, t\rangle$}\\{$\lambda y. \ Loves(y, b)$} [. V\\{$\langle e, \langle e, t\rangle\rangle$}\\{$\lambda x. \lambda y. \ Loves(y, x)$}  [. \textbf{\textit{loves}} ]] [. DP\\{$e$}\\{$b$} [. \textbf{\textit{Bj\"{o}rn}} ] ]]]
\end{tikzpicture}

\end{multicols}








Now let's do the following together: 

Every singer is Swedish: 

\textcolor{blue}{
\begin{tikzpicture}	
[sibling distance=20pt, level distance=70pt][scale=1]
\Tree [. S\\$t$\\{$\forall x.  \ [Singer(x) \rightarrow Swedish(x)]$} [. DP\\{$\langle\langle e, t\rangle, t\rangle$}\\{$\lambda Q. \ \forall x.  \ [Singer(x) \rightarrow Q(x)]$} [. D\\{$\langle\langle e, t\rangle, \langle \langle e, t\rangle, t \rangle\rangle$}\\{$\lambda P. \ \lambda Q. \ \forall x.  \ [P(x) \rightarrow Q(x)]$} [. \textit{\textbf{every}} ] ] [. NP\\{$\langle e, t\rangle$}\\{$\lambda x. \ Singer(x)$} [. \textit{\textbf{singer}} ] ]]  [. VP\\{$\langle e, t\rangle$}\\{$\lambda x. \ Swedish(x)$} [. V\\{$\langle\langle e, t\rangle, \langle e, t\rangle\rangle$}\\{$\lambda P. \ P$} [.  \textit{\textbf{is}} ]] [. AdjP\\{$\langle e, t\rangle$}\\{$\lambda x. \ Swedish(x)$} [. \textbf{\textit{Swedish}} ]]]]]
\end{tikzpicture}}

\newpage
Benny is a musician: 


\textcolor{blue}{
\begin{tikzpicture}	
[sibling distance=20pt, level distance=70pt][scale=1]
\Tree [. S\\{$t$}\\{$Musician(a)$} [. DP\\{$e$}\\{$a$} [. \textbf{\textit{Agnetha}}   ]] [. VP\\{$\langle e, t\rangle$}\\{$\lambda x. \ Musician(x)$} [. V\\{$\langle\langle e, t\rangle, \langle e, t\rangle\rangle$}\\{$\lambda P. \ P$} [.  \textit{\textbf{is}} ]] [. DP\\{$\langle e, t\rangle$}\\{$\lambda x. \ Musician(x)$} [. D\\{$\langle\langle e, t\rangle, \langle e, t\rangle\rangle$}\\{$\lambda P. \ P$} [.  \textit{\textbf{a}} ]] [. NP\\{$\langle e, t\rangle$}\\{$\lambda x. \ Musician(x)$} [. \textbf{\textit{musician}} ]]]]]
\end{tikzpicture}}


\newpage

Agnetha is with Bj\"{o}rn: 


\textcolor{blue}{
\begin{tikzpicture}	
[sibling distance=20pt, level distance=70pt][scale=1]
\Tree [. S\\{$t$}\\{$With(a, b)$} [. DP\\{$e$}\\{$a$} [. \textbf{\textit{Agnetha}}   ]] [. VP\\{$\langle e, t\rangle$}\\{$\lambda y.  \ With(y, b)$} [. V\\{$\langle\langle e, t\rangle, \langle e, t\rangle\rangle$}\\{$\lambda P. \ P$} [.  \textit{\textbf{is}} ]] [. PP\\{$\langle e, t\rangle$}\\{$\lambda y.  \ With(y, b)$} [. P\\{$\langle e, \langle e, t\rangle\rangle$}\\{$\lambda x. \lambda y.  \ With(y, x)$} [.  \textit{\textbf{with}} ]] [. DP\\{$e$}\\{$b$} [. \textbf{\textit{Bj\"{o}rn}}   ]]]]]
\end{tikzpicture}}

\newpage

Agnetha is proud of Bj\"{o}rn: 

\textcolor{blue}{
\begin{tikzpicture}	
[sibling distance=20pt, level distance=70pt][scale=1]
\Tree [. S\\{$t$}\\{$Proud(a, b)$} [. DP\\{$e$}\\{$a$} [. \textbf{\textit{Agnetha}}   ]] [. VP\\{$\langle e, t\rangle$}\\{$\lambda y.  \ Proud(y, b)$} [. V\\{$\langle\langle e, t\rangle, \langle e, t\rangle\rangle$}\\{$\lambda P. \ P$} [.  \textit{\textbf{is}} ]] [. AdjP\\{$\langle e, t\rangle$}\\{$\lambda y.  \ Proud(y, b)$} [. Adj\\{$\langle e, \langle e, t\rangle\rangle$}\\{$\lambda x. \lambda y.  \ Proud(y, x)$} [.  \textit{\textbf{proud}} ]] [. PP\\{$e$}\\{$b$} [. P\\{$\langle e, e\rangle$}\\{$\lambda x. x$} [. \textbf{\textit{of}} ] ] [. DP\\{$e$}\\{$b$} [. \textbf{\textit{Bj\"{o}rn}}   ]]]]]]
\end{tikzpicture}}


\newpage

Benny is not a singer.

\textcolor{blue}{
\begin{tikzpicture}	
[sibling distance=20pt, level distance=70pt][scale=1]
\Tree [. S\\{$t$}\\{$\neg Singer(a)$} [. DP\\{$e$}\\{$a$} [. \textbf{\textit{Agnetha}}   ]] [. VP\\{$\langle e, t\rangle$}\\{$\lambda x. \ \neg Singer(x)$} [. V\\{$\langle\langle e, t\rangle, \langle e, t\rangle\rangle$}\\{$\lambda P. \ P$} [.  \textit{\textbf{is}} ]] [. NegP\\{$\langle e, t\rangle$}\\{$\lambda x. \ \neg Singer(x)$} [. Neg\\{$\langle\langle e, t\rangle, \langle e, t\rangle\rangle$}\\{$\lambda P. \lambda x. \ \neg P(x)$} [.  \textit{\textbf{not}} ] ] [. DP\\{$\langle e, t\rangle$}\\{$\lambda x. \ Singer(x)$} [. D\\{$\langle\langle e, t\rangle, \langle e, t\rangle\rangle$}\\{$\lambda P. \ P$} [.  \textit{\textbf{a}} ]] [. NP\\{$\langle e, t\rangle$}\\{$\lambda x. \ Musician(x)$} [. \textbf{\textit{singer}} ]]]]]]
\end{tikzpicture}}

\newpage


\textbf{EXTRA EXAMPLES} 

Some singer is Swedish: 

\begin{tikzpicture}	
[sibling distance=20pt, level distance=70pt][scale=1]
\Tree [. S\\$t$\\{$\exists x.  \ [Singer(x) \wedge Swedish(x)]$} [. DP\\{$\langle\langle e, t\rangle, t\rangle$}\\{$\lambda Q. \ \exists x.  \ [Singer(x) \wedge Q(x)]$} [. D\\{$\langle\langle e, t\rangle, \langle \langle e, t\rangle, t \rangle\rangle$}\\{$\lambda P. \ \lambda Q. \ \exists x.  \ [P(x) \wedge Q(x)]$} [. \textit{\textbf{some}} ] ] [. NP\\{$\langle e, t\rangle$}\\{$\lambda x. \ Singer(x)$} [. \textit{\textbf{singer}} ] ]]  [. VP\\{$\langle e, t\rangle$}\\{$\lambda x. \ Swedish(x)$} [. V\\{$\langle\langle e, t\rangle, \langle e, t\rangle\rangle$}\\{$\lambda P. \ P$} [.  \textit{\textbf{is}} ]] [. AdjP\\{$\langle e, t\rangle$}\\{$\lambda x. \ Swedish(x)$} [. \textbf{\textit{Swedish}} ]]]]]
\end{tikzpicture}


Some singer is not Swedish: 

\begin{tikzpicture}	
[sibling distance=20pt, level distance=70pt][scale=1]
\Tree [. S\\$t$\\{$\exists x.  \ [Singer(x) \wedge \neg Swedish(x)]$} [. DP\\{$\langle\langle e, t\rangle, t\rangle$}\\{$\lambda Q. \ \exists x.  \ [Singer(x) \wedge Q(x)]$} [. D\\{$\langle\langle e, t\rangle, \langle \langle e, t\rangle, t \rangle\rangle$}\\{$\lambda P. \ \lambda Q. \ \exists x.  \ [P(x) \wedge Q(x)]$} [. \textit{\textbf{some}} ] ] [. NP\\{$\langle e, t\rangle$}\\{$\lambda x. \ Singer(x)$} [. \textit{\textbf{singer}} ] ]]  [. VP\\{$\langle e, t\rangle$}\\{$\lambda x. \ \neg Swedish(x)$} [. V\\{$\langle\langle e, t\rangle, \langle e, t\rangle\rangle$}\\{$\lambda P. \ P$} [.  \textit{\textbf{is}} ]] [. NegP\\{$\langle e, t\rangle$}\\{$\lambda x. \ \neg Swedish(x)$} [. Neg\\{$\langle\langle e, t\rangle, \langle e, t\rangle\rangle$}\\{$\lambda P. \lambda x. \ \neg P(x)$} [.  \textit{\textbf{not}} ] ] [. AdjP\\{$\langle e, t\rangle$}\\{$\lambda x. \ Swedish(x)$} [. \textbf{\textit{Swedish}} ]]]]]
\end{tikzpicture}


\newpage
Benny is curious about Frida: 


\begin{tikzpicture}	
[sibling distance=20pt, level distance=70pt][scale=1]
\Tree [. S\\{$t$}\\{$Curious(e, f)$} [. DP\\{$e$}\\{$e$} [. \textbf{\textit{Benny}}   ]] [. VP\\{$\langle e, t\rangle$}\\{$\lambda y.  \ Curious(y, f)$} [. V\\{$\langle\langle e, t\rangle, \langle e, t\rangle\rangle$}\\{$\lambda P. \ P$} [.  \textit{\textbf{is}} ]] [. AdjP\\{$\langle e, t\rangle$}\\{$\lambda y.  \ Curious(y, f)$} [. Adj\\{$\langle e, \langle e, t\rangle\rangle$}\\{$\lambda x. \lambda y.  \ Curious(y, x)$} [.  \textit{\textbf{curious}} ]] [. PP\\{$e$}\\{$f$} [. P\\{$\langle e, e\rangle$}\\{$\lambda x. x$} [. \textbf{\textit{about}} ] ] [. DP\\{$e$}\\{$f$} [. \textbf{\textit{Frida}}   ]]]]]]
\end{tikzpicture}


\end{document}